\documentclass[]{scrartcl}

\usepackage{amsmath}
\usepackage{amssymb}
\usepackage[utf8]{inputenc}
\usepackage[T1]{fontenc}
\usepackage{lmodern}
\usepackage{ngerman}
\usepackage{geometry}
\usepackage{graphicx}
\usepackage{wrapfig}
\usepackage{caption}
\usepackage{wasysym}
\usepackage{siunitx}
\usepackage{picinpar}
\usepackage{tikz}
\usepackage{float}

\renewcommand{\figurename}{Abb.}
\usepackage[
	colorlinks=true,
	urlcolor=blue,
	linkcolor=black
]{hyperref}


%Hier Titel und so
\newcommand{\versuchnummer}{V59} 
\newcommand{\versuchname}{Modulation und Demodulation elektrischer Schwingungen} 
\newcommand{\versuchdatum}{24.02.16} 


\title{Versuch \versuchnummer\\ \versuchname}
\subtitle{Physikalisches Fortgeschrittenenpraktikum}
\author{Robert Rauter und Björn Lindhauer}
\date{\versuchdatum} 
\begin{document}
\begin{titlepage}
{\large \versuchdatum}
\vspace{7cm}
\begin{center}
\textbf{\huge Versuch \versuchnummer:}\\
\vspace{0.5cm}
\textbf{\huge \versuchname}\\
\vspace{0.2cm}
\textbf{ Physikalisches Fortgeschrittenenpraktikum}\\
\vspace{9cm}

{\Large Robert Rauter \ \ \hspace{1.5cm} und \hspace{1.5cm} Björn Lindhauer}\\
{ \url{robert.rauter@tu-dortmund.de} \ \ \hspace{2cm} \url{bjoern.lindhauer@tu-dortmund.de}}
\end{center}
\end{titlepage}
\section{Einleitung}
Unter Modulation wird Veränderung der Amplitude, der Phase oder der Frequenz einer Welle im Rhythmus des Nachrichtensignals verstanden. Sie wird benötigt, um Signale mit elektromagnetische Wellen zu übertragen.\\
Das übertragene Signal muss beim Empfänger anschließend zurück gewonnen werden. Dieser Vorgang wird als Demodulation bezeichnet.\\ Mit der Zeit wurden verschiedene Verfahren mit unterschiedlichen Stärken und Schwächen in der Störsicherheit, im Wirkungsgrad, in der Verzerrungsfreiheit und in der Breite des Frequenzspektrums entwickelt. Diese Verfahren lassen sich in zwei Klassen, den Amplitudenmodulations- und Phasenwinkelmodulation- Verfahren unterteilen.\\
In diesen Versuch sollen Beispiele beider Verfahrensklassen untersucht werden.
\section{Theoretische Grundlagen}
\subsection{Amplitudenmodulation}
Eine Amplitudenmodulation kann durch eine hochfrequente Trägerschwingung $U_{\text{T}}\left(t\right)$ mit niederfrequenten Modulationssignal $U_{\text{M}}\left(t\right)$ erreicht werden.\\
Die zeitliche Entwicklung der Amplituden lassen sich durch
\begin{align}
U_{\text{T}}\left(t\right)=\hat{U}_{\text{T}} \cos \omega_{\text{T}}t  \text{ und }U_{\text{M}}\left(t\right)=\hat{U}_{\text{M}} \cos \omega_{\text{M}}t
\end{align}
darstellen. Dabei sind die $\omega_{i}$ jeweils die Frequenzen und $\hat{U}_{i}$ die Amplituden der Trägerschwingung bzw. der Modulationsschwingung.\\
Daraus folgt für die amplitudenmodulierte Schwingung
\begin{align}
U_3\left(t\right)=\hat{U}_{\text{T}} \left(1+m\cos \omega_{\text{M}}t\right)\cos \omega_{\text{T}}t
\label{eq:amplitudederamplitudenmoduliertenschwingung}
\end{align}
mit den Modulationsgrad genannten Größe
\begin{align}
m=\gamma \hat{U}_{\text{M}} \hspace{0.5cm}\text{.}
\end{align}
Der Modulationsgrad kann nur Werte zwischen 0 und 1 annehmen, sodass die Maxima der modulierten Schwingung zwischen $\hat{U}_{\text{T}}\left(1-m\right)$ und $\hat{U}_{\text{T}}\left(1+m\right)$ liegen, wie in Abbildung 
\ref{fig:zeitabhaengigkeit_momentanspannung} anschaulich dargestellt.
Die Gleichung \ref{eq:amplitudederamplitudenmoduliertenschwingung} lässt sich durch trigonometrischer Beziehungen in die Form
\begin{align}
U_3\left(t\right)=\hat{U}_{\text{T}}\left(\cos \omega_{\text{T}}t+ \frac{m}{2}\cos \left( \omega_{\text{T}}+\omega_{\text{M}}\right)t +\frac{m}{2}\cos \left( \omega_{\text{T}}-\omega_{\text{M}}\right)t  \right) 
\label{eq:amplitudederamplitudenmoduliertenschwingungbesser}
\end{align}
überführen, anhand jener das Frequenzspektrum einfach abgelesen werden kann. Das Spektrum besteht aus 3 Linien bei den Frequenzen $\omega_{\text{T}}$, $\omega_{\text{T}}+\omega_{\text{M}}$ und $\omega_{\text{T}}-\omega_{\text{M}}$ und ist Beispielhaft in Abbildung \ref{fig:frequenzsprektum_amplitudenmodulierten} dargestellt.\\
Besteht die Modulationsspannung $\hat{U}_\text{M}$ aus einer Reihe von verschiedener Frequenzen, so verbreitern sich die beiden äußeren Linien zu Frequenzbändern.\\
\begin{figure}[H]
\centering 
\includegraphics[width=10cm]{images/zeitabhaengigkeit_momentanspannung.png}
\caption{Zeitliche Darstellung der Momentanspannung $U_3$ nach Gleichung \ref{eq:amplitudederamplitudenmoduliertenschwingung} (1)}
\label{fig:zeitabhaengigkeit_momentanspannung}
\end{figure}
In Gleichung \ref{eq:amplitudederamplitudenmoduliertenschwingungbesser} ist ersichtlich, dass das Signal bei $\omega_\text{T}$ keinerlei Informationen besitzt, da sie unabhängig von $\hat{U}_\text{M}$ ist. Es wird auch als Trägerabstrahlung bezeichnet und stellt in der Praxis nur einen unnötigen Energieverbrauch da und wird deswegen vermieden.\\
Es kann außerdem Energie gespart werden, indem eines der beiden Bänder durch ein Bandfilter unterdrückt wird. Dies wird auch Einseitenbandmodulation genannt. Es ist möglich, da die beiden Bänder die selbe Information enthalten.
\begin{figure}[H]
\centering 
\includegraphics[width=10cm]{images/frequenzsprektum_amplitudenmodulierten.png}
\caption{Frequenzspektrum einer nach \ref{eq:amplitudederamplitudenmoduliertenschwingungbesser} amplitudenmodulierten Schwingung (1)}
\label{fig:frequenzsprektum_amplitudenmodulierten}
\end{figure}
Die Amplitudenmodulation besitzt jedoch eine geringe Störsicherheit und Verzerrungsfreiheit.
\subsection{Frequenzmodulation}\label{sec:frequenzmodulation}
Bei der Frequenzmodulation wird nicht die Amplitude, sondern die Frequenz im Rhythmus des Modulationssignales geändert.\\
Es ergibt sich ein Signal der Form
\begin{align}
U\left(t\right)=\hat{U}\sin \left( \omega_{\text{T}}t+ m \frac{\omega_{\text{T}}}{\omega_{\text{M}}}\cos \omega_{\text{M}}t \right)\label{eq:frequenzmodulationsignal}
\end{align}
mit den Modulationsgrad $m$, der durch die Modulationsspannung gegeben ist.\\
Durch Ableitung des Arguments der Sinus-Funktion in \ref{eq:frequenzmodulationsignal} kann die Momentanfrequenz
\begin{align}
f\left(t\right)=\frac{\omega_{\text{T}}}{2\pi}\left(1+m\sin \omega_{\text{M}}t\right)
\end{align} 
bestimmt werden.\\
Des weiteren wird die Größe $m\frac{\omega_{\text{T}}}{2\pi}$ als Frequenzhub bezeichnet und gibt die Variationsbreite der Schwingungsfrequenz an.\\
Wie ein frequenzmoduliertes Signal aussehen könnte, ist in Abbildung
\ref{fig:frequenzmodulation_beispiel} abgebildet.
\begin{figure}[H]
\centering 
\includegraphics[width=13cm]{images/frequenzmodulation_bsp.png}
\caption{Beispielhafter zeitlicher Verlauf eines frequenzmodulierten Signals (1)}
\label{fig:frequenzmodulation_beispiel}
\end{figure} 
Es soll im folgenden nur die Schmalband-Frequenzmodulation betrachtet werden, mit niedrigen Frequenzhub, sodass
\begin{align}
m\frac{\omega_{\text{T}}}{\omega_{\text{M}}}\ll 1 \label{eq:frequenzmodulationsignalentwicklungsparameter}
\end{align}
ist und \ref{eq:frequenzmodulationsignal} nach dieser Größe entwickelt werden kann. Dazu wird \ref{eq:frequenzmodulationsignal} zunächst durch Additionstheoreme in die Form
\begin{align}
U\left(t\right)=\hat{U}\left(\sin   \omega_{\text{T}}t \cos\left(m \frac{\omega_{\text{T}}}{\omega_{\text{M}}}\cos \omega_{\text{M}}t \right)+\cos   \omega_{\text{T}}t \sin\left(m \frac{\omega_{\text{T}}}{\omega_{\text{M}}}\cos \omega_{\text{M}}t \right) \right)\label{eq:frequenzmodulationsignalentwicklungsform}
\end{align}
gebracht, da diese sich leichter nach \ref{eq:frequenzmodulationsignalentwicklungsparameter} entwickeln lässt.\\
Die entstehende Gleichung
\begin{align}
U\left(t\right)=\hat{U}\left(\sin   \omega_{\text{T}}t+ \frac{m}{2}\frac{\omega_{\text{T}}}{\omega_{\text{M}}}\cos\left(\omega_{\text{T}}+\omega_{\text{M}}\right) +\frac{m}{2}\frac{\omega_{\text{T}}}{\omega_{\text{M}}}\cos\left(\omega_{\text{T}}-\omega_{\text{M}}\right)  \right)
\end{align}
hat die gleiche Form wie die Gleichung \ref{eq:amplitudederamplitudenmoduliertenschwingungbesser} beim amplitudenmodulierte Signal und besteht auch aus drei Teilschwingungen, jedoch sind die Seitenlinien um $90^\circ$ gegen die Trägerschwingung in der Phase verschoben.\\
Bei starker Frequenzmodulation mit
\begin{align}
m \omega_{\text{T}} \approx \omega_{\text{M}}
\end{align}
muss die Entwicklung von \ref{eq:frequenzmodulationsignalentwicklungsform} um höhere Ordnungen ergänzt werden.\\
Dabei ergibt sich die Darstellung
\begin{align}
U\left(t\right)=\hat{U}\sum\limits_{n=-\infty}^{\infty} J_n\left(m\frac{\omega_{\text{T}}}{\omega_{\text{M}}} \right)\sin \left(\omega_{\text{T}}+n\omega_{\text{M}}\right)t\label{eq:frequenzmodulation_bessel}
\end{align}
mit der Besselsche Funktion $J_n\left(x\right)$. Dabei wird das Frequenzspektrum in Falle eines hohen Modulationsgrades vollständig abgedeckt, wie es aus Gleichung \ref{eq:frequenzmodulation_bessel} hervorgeht. In praktischen Anwendungen müssen jedoch nur Frequenzen nahe $\omega_{\text{T}}$ betrachtet werden, da die Besselsche Funktion für $x\l 1$ schnell abfällt.   
\subsection{Modulationsschaltungen}
\subsubsection*{Amplitudenmodulation}
Um eine Amplitudenmodulation zu erreichen, muss das Produkt zweier Spannungen gebildet werden. Dafür wird ein Bauteil mit nicht-linearer Kennlinie benötigt.\\
Die einfachste Modulationsschaltung besteht aus einer Diode und den zwei Spannungsquellen und ist in Abbildung \ref{fig:einfachemodulation} skizziert.
\begin{figure}[H]
\centering 
\includegraphics[width=8cm]{images/primitive_modulationsschaltung.png}
\caption{Einfache Modulationsschaltung mit einer Diode (1)}
\label{fig:einfachemodulation}
\end{figure} 
Die Amplitudenmodulation dieser Schaltung wird ersichtlich, wenn der Strom
\begin{align}
I\left(U\right)=a_0+a_1 U+a_2U^2 + O(U^3)
\end{align}
nach $U$ entwickelt wird und die Summe der Träger- und Modulationsspannung in dieser Entwicklung eingesetzt werden:
\begin{align}
I\left(U_{\text{T}}+U_{\text{M}}\right)=a_0+a_1\left(U_{\text{T}}+U_{\text{M}}\right)+a_2\left(U_{\text{T}}^2+U_{\text{M}}^2+2U_{\text{T}}U_{\text{M}}\right) + O(\left(U_{\text{T}}+U_{\text{M}}\right)^3)
\end{align}
Es tauchen jedoch neben den gewünschten Term $U_{\text{T}}U_{\text{M}}$ noch weitere Terme auf, deren Frequenzen aber normalerweise weit außerhalb des Frequenzbandes von $\omega_{\text{T}}-\omega_{\text{M}}$ bis $\omega_{\text{T}}+\omega_{\text{M}}$ liegt und können somit durch einen Bandfilter unterdrückt werden.\\
Dieses Verfahren ist unökonomisch, denn es sollte vermieden werden unnötige Beiträge in $I$ zu erzeugen, da diese nur Energie kosten.\\
Die Beiträge können auf verschiedene Arten vermieden werden. Zum einen können andere Bauteile verwendet werden, bei denen die höheren Glieder der Entwicklung verschwinden. Zum anderen können die Bauteile so angeordnet werden, dass sich störende Glieder herausheben.\\
Ein Beispiel für die letztere Methode ist der Ringmodulator.\\
Ein Ringmodulator besteht aus vier zusammengeschalteten Dioden, wie in Abbildung \ref{fig:ringmodulator} dargestellt.\\
Die Diodenzweige A und B sowie C und D dienen als Spannungsteiler für die Trägerspannung $U_{\text{T}}$, die an Eingang L angelegt wird.\\
Die geteilten Spannungen werden an $\alpha$ und $\beta$ abgegriffen und über Hochfrequenz-Transformator an den Ausgang R ausgegeben.\\
Wenn alle Dioden gleiche elektrische Eigenschaften haben, ändert sich zwar die Last der einzelnen Dioden, das Teilungsverhältnis ändert sich jedoch nicht, sodass keine Spannung zwischen $\alpha$ und $\beta$ anliegt.\\
Wird jedoch eine Modulationsspannung am Eingang X angelegt, so ändern sich die Teilverhältnisse und zwischen $\alpha$ und $\beta$ ist eine Spannung im Rhythmus von $U_{\text{M}}\left(t\right)$ ab zugreifen.\\
\begin{figure}[H]
\centering 
\includegraphics[width=13cm]{images/ringmodulatorschaltung.png}
\caption{Schaltbild eines Ringmodulators (1)}
\label{fig:ringmodulator}
\end{figure} 
Bei idealen Verhältnissen ist
\begin{align}
U_{\text{R}}=\gamma U_{\text{T}}U_{\text{M}}\label{eq:ringmodulatorprodukt}
\end{align}
nur proportional zu den Produkt aus $U_{\text{T}}$ und $U_{\text{M}}$ und es treten nur die beiden Seitenlinien im Frequenzspektrum auf.
\subsubsection*{Frequenzmodulation}
Im folgenden soll ein Frequenzmodulator mit geringem Frequenzhub beschrieben werden.\\
Die Voraussetzungen für ein frequenzmoduliertes Signal wurde im Abschnitt \ref{sec:frequenzmodulation} diskutiert. Es wird jeweils eine amplitudenmodulierte Seitenlinien mit Frequenz $\omega_{\text{T}}-\omega_{\text{M}}$ und $\omega_{\text{T}}+\omega_{\text{M}}$ sowie ein um $90^\circ$ verschobenes Trägersignal benötigt.\\
Im vorherigen Abschnitt wurde mit den Ringmodulator eine Möglichkeit beschrieben, um verlustfrei ein amplitudenmoduliertes Signal mit den benötigten Seitenlinien zu erzeugen. Dieses muss danach noch um den um $90^\circ$ verschobenen Trägersignal ergänzt werden. Dies ist mit der in Abbildung \ref{fig:frequenzmodulationsschaltung} dargestellten Schaltung möglich.
\begin{figure}[H]
\centering 
\includegraphics[width=13cm]{images/frequenzmodulationsschaltung.png}
\caption{Schaltbild für einen Frequenzmodulator mit geringem Frequenzhub (1)}
\label{fig:frequenzmodulationsschaltung}
\end{figure} 
\subsection{Demodulator-Schaltungen}
\subsubsection*{Amplitudenmodulierte Signale}
Aus Gleichung \ref{eq:ringmodulatorprodukt} geht hervor, dass ein Ringmodulator das Produkt zweier Spannungen bildet und dabei treten Frequenzen mit $\omega_{\text{T}}-\omega_{\text{M}}$ und $\omega_{\text{T}}+\omega_{\text{M}}$ auf.\\
Um eine Demodulation zu erhalten, wird auf L eine Spannung mit der Frequenz $\omega_{\text{T}}$ angelegt, sodass am Ausgang X eine Spannung mit Frequenzen $\omega_{\text{M}}$, $2\omega_{\text{T}}-\omega_{\text{M}}$ und $2\omega_{\text{T}}+\omega_{\text{M}}$ anliegt. Die störenden Frequenzen können mithilfe eines Tiefpasses herausgefiltert werden, da meistens $\omega_{\text{T}}$ viel größer als $\omega_{\text{M}}$ ist. Nach den Tiefpass kann anschließend ein Signal mit der gewünschten Frequenz $\omega_{\text{M}}$ abgegriffen werden. Der Schaltplan ist in Abbildung \ref{fig:demodulationsschaltungmitringmodulator} dargestellt.\\
Ein Problem dabei ist, dass die Trägerfrequenz für die Demodulation starr mit der Trägerfrequenz der Modulation gekoppelt sein muss. Das Problem kann durch sogenannte Phasenregelkreise umgangen werden.
\begin{figure}[H]
\centering 
\includegraphics[width=10cm]{images/demodulationsschaltungmitringmodulator.png}
\caption{Demodulator-Schaltung mit einem Ringmodulator (1)}
\label{fig:demodulationsschaltungmitringmodulator}
\end{figure} 
Alternativ kann auch eine Diode zur Demodulation verwendet werden, die nach Abbildung \ref{fig:demodulationsschaltungmitgleichrichtdiode} aufgebaut wird.
\begin{figure}[H]
\centering 
\includegraphics[width=10cm]{images/demodulationsschaltungmitgleichrichtdiode.png}
\caption{Demodulator-Schaltung mit einer Gleichrichter-Diode (1)}
\label{fig:demodulationsschaltungmitgleichrichtdiode}
\end{figure}
Die Diode schneidet sämtliche negative Halbwellen ab, sodass ein Signal der Form wie in der Abbildung \ref{fig:AusgangsspannungDemodulationDiode}.
\begin{figure}[H]
\centering 
\includegraphics[width=14cm]{images/AusgangsspannungDemodulationDiode.png}
\caption{Demodulator-Schaltung mit einer Gleichrichter-Diode (1)}
\label{fig:AusgangsspannungDemodulationDiode}
\end{figure}
\section{Durchführung}

\section{Auswertung}

\subsection{Amplitudenmodulation mit Ringmodulator}
Mit einem Ringmodulator wird ein amplitudenmoduliertes Signal mit Trägerunterdrückung erzeugt. Freequenzen f und Amplituden U von Träger- und Modulationssignal lauten:
\begin{enumerate}
	\item Trägersignal f$_t10,0$\,MHz
	\item Modulationssignal f$_m=158$\,kHz
\end{enumerate}
Der Ergebnis der Amplitudenmodulation ist in Abbildung \ref{fig:ampmodring} dargestellt.
\begin{center}
	\includegraphics[width=10cm]{images/ampmodring.png}
	\captionof{figure}{Amplitudenmoduliertes Signal, erzeugt mit Ringmodulator}
	\label{fig:ampmodring}
\end{center}
Eine Aufnahme des Frequenzspektrums, welches mit einem Frequenzanalysators aufgenommen wurde, ist in Abbildung \ref{fig:freqampmodring} dargestellt. Der Anteil der Trägerfrequenz ist um zwei Dekaden geringer als die der Seitenbänder. 
\begin{center}
	\includegraphics[width=10cm]{images/freqampmodring.png}
	\captionof{figure}{Frequenzspektrum eines amplitudenmoduliertes Signals mit einem Ringmodulator, die Trägerfrequenz ist um eine Dekade unterdrückt.}
	\label{fig:freqampmodring}
\end{center}

\subsection{Ausmessung des phasenempfindlichen Gleichrichters}
Bevor die Amplitudendemodulation vorgenommen wird, wird zunächst der phasenempfindliche Gleichrichter ausgemessen und untersucht, ob die Ausgangsspannung der Relation
\begin{align}
U_{\text{ph}}=a\cos(\phi+b)
\label{eq:phasen}
\end{align}
folgt, wobei $\phi$ dabei die Phasendifferenz der beiden Eingangssignale darstellt. \\\
Die Phasendifferenzen $\phi$ werden über eine variable Verzögerungsleitung eingestellt, wodurch sich die Phasendifferenzen $\phi$ aus 
\begin{align}
\phi = tf2\pi
\end{align}
ergibt, mit f der Frequenz und t der Verzögerungszeit. Dazu wird die Frequenz auf f=9.58\si{MHz} eingestellt, um ein möglichst großes Intervall für die Phasendifferenz abzudecken. Die aufgenommenen Messwerte sind in Tabelle \ref{tab:phasen} dargestellt. \\
\begin{center}
	\begin{tabular}{|c|c||c|c|}
		\hline $\Delta$t [ns] & U [V] & $\Delta$t [ns] & U [V]\\
		\hline	0	&	-0,247	&	32	&	0,17	\\
		\hline	1	&	-0,239	&	40	&	0,238	\\
		\hline	2	&	-0,231	&	48	&	0,263	\\
		\hline	4	&	-0,215	&	56	&	0,21	\\
		\hline	8	&	-0,175	&	64	&	0,127	\\
		\hline	16	&	-0,072	&	72	&	0,003	\\
		\hline	18	&	-0,038	&	80	&	-0,114	\\
		\hline	19	&	-0,02	&	88	&	-0,205	\\
		\hline	20	&	-0,006	&	96	&	-0,261	\\
		\hline	24	&	0,061	&	104	&	-0,242	\\
		\hline	28	&	0,117	&	112	&	-0,17	\\
		\hline	30	&	0,143	&		&	\\	
		\hline
	\end{tabular}
	\captionof{table}{Aufgenommene Messwerte für Verzögerungszeit $\Delta$t und Spannung zur Vermessung des phasenempfindlichen Gleichrichters}
	\label{tab:phasen}
\end{center}
Mithilfe von Python werden die Verzögerungszeiten in Phasendifferenzen umgerechnet und eine nicht-lineare Ausgleichsrechnung durchgeführt, mithilfe von Gleichung \ref{eq:phasen}. Der entsprechende Plot ist in Abbildung \ref{fig:phasen} dargestellt, die nicht-lineare Ausgleichsrechnung ergibt dabei die folgenden Parameter:
\begin{align}
a &= (-0.259 \pm 0.001) V \\
b &= ( 0.352 \pm 0.004) 
\end{align}
\begin{center}
	\includegraphics[width=10cm]{images/plotgleich.png}
	\captionof{figure}{Messwerte für den phasenabhängigen Gleichrichter und Cosinus-Fit für diese.}
	\label{fig:phasen}
\end{center}
Es zeigt sich ein Cosinus-Verhalten der Ausgangsspannung in Abhängigkeit von der Phasendifferenz.
\subsection{Amplitudendemodulation mit Ringmodulator}


\subsection{Amplitudenmodulation mit Diode}

\subsection{Amplitudendemodulation mit Diode}

\subsection{Frequenzmodulation}

\subsection{Frequenzdemodulation}

\section{Diskussion}

\section{Quellen}
(1) Versuchsbeschreibung
\end{document}