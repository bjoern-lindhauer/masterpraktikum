\documentclass[]{scrartcl}

\usepackage{amsmath}
\usepackage{amssymb}
\usepackage[utf8]{inputenc}
\usepackage[T1]{fontenc}
\usepackage{lmodern}
\usepackage{ngerman}
\usepackage{geometry}
\usepackage{graphicx}
\usepackage{wrapfig}
\usepackage{caption}
\usepackage{wasysym}
\usepackage{siunitx}
\usepackage{picinpar}
\usepackage{tikz}
\usepackage{float}

\renewcommand{\figurename}{Abb.}
\usepackage[
	colorlinks=true,
	urlcolor=blue,
	linkcolor=black
]{hyperref}


%Hier Titel und so
\newcommand{\versuchnummer}{Nummer} 
\newcommand{\versuchname}{Name} 
\newcommand{\versuchdatum}{Datum} 


\title{Versuch \versuchnummer\\ \versuchname}
\subtitle{Physikalisches Fortgeschrittenenpraktikum}
\author{Robert Rauter und Björn Lindhauer}
\date{\versuchdatum} 
\begin{document}
\begin{titlepage}
{\large \versuchdatum}
\vspace{7cm}
\begin{center}
\textbf{\huge Versuch \versuchnummer}{V56}\\
\vspace{0.5cm}
\textbf{\huge \versuchname}{Modulation und Demodulation elektrischer Schwingungen}\\
\vspace{0.2cm}
\textbf{ Physikalisches Fortgeschrittenenpraktikum}\\
\vspace{9cm}

{\Large Robert Rauter \ \ \hspace{1.5cm} und \hspace{1.5cm} Björn Lindhauer}\\
{ \url{robert.rauter@tu-dortmund.de} \ \ \hspace{2cm} \url{bjoern.lindhauer@tu-dortmund.de}}
\end{center}
\end{titlepage}
\section{Einleitung}

\section{Theoretische Grundlagen}

\section{Durchführung}

\section{Auswertung}

\subsection{Amplitudenmodulation mit Ringmodulator}
Mit einem Ringmodulator wird ein amplitudenmoduliertes SignaL mit Trägerunterdrückung erzeugt. Freequenzen f und Amplituden U von Träger- und Modulationssignal lauten:
\begin{enumerate}
	\item Trägersignal
	\item Modulationssignal 
\end{enumerate}
\subsection{Amplitudendemodulation mit Ringmodulator}

\subsection{Amplitudenmodulation mit Diode}

\subsection{Amplitudendemodulation mit Diode}

\subsection{Frequenzmodulation}

\subsection{Frequenzdemodulation}

\section{Diskussion}

\section{Quellen}

\end{document}