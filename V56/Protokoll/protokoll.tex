\documentclass[]{scrartcl}

\usepackage{amsmath}
\usepackage{amssymb}
\usepackage[utf8]{inputenc}
\usepackage[T1]{fontenc}
\usepackage{lmodern}
\usepackage{ngerman}
\usepackage{geometry}
\usepackage{graphicx}
\usepackage{wrapfig}
\usepackage{caption}
\usepackage{wasysym}
\usepackage{siunitx}
\usepackage{picinpar}
\usepackage{tikz}
\usepackage{float}

\renewcommand{\figurename}{Abb.}
\usepackage[
	colorlinks=true,
	urlcolor=blue,
	linkcolor=black
]{hyperref}


%Hier Titel und so
\newcommand{\versuchnummer}{V59} 
\newcommand{\versuchname}{Modulation und Demodulation elektrischer Schwingungen} 
\newcommand{\versuchdatum}{24.02.16} 


\title{Versuch \versuchnummer\\ \versuchname}
\subtitle{Physikalisches Fortgeschrittenenpraktikum}
\author{Robert Rauter und Björn Lindhauer}
\date{\versuchdatum} 
\begin{document}
\begin{titlepage}
{\large \versuchdatum}
\vspace{7cm}
\begin{center}
\textbf{\huge Versuch \versuchnummer:}\\
\vspace{0.5cm}
\textbf{\huge \versuchname}\\
\vspace{0.2cm}
\textbf{ Physikalisches Fortgeschrittenenpraktikum}\\
\vspace{9cm}

{\Large Robert Rauter \ \ \hspace{1.5cm} und \hspace{1.5cm} Björn Lindhauer}\\
{ \url{robert.rauter@tu-dortmund.de} \ \ \hspace{2cm} \url{bjoern.lindhauer@tu-dortmund.de}}
\end{center}
\end{titlepage}
\section{Einleitung}
Unter Modulation wird Veränderung der Amplitude, der Phase oder der Frequenz einer Welle im Rhythmus des Nachrichtensignals verstanden. Sie wird benötigt, um Signale mit elektromagnetische Wellen zu übertragen.\\
Das übertragene Signal muss beim Empfänger anschließend zurück gewonnen werden. Dieser Vorgang wird als Demodulation bezeichnet.\\ Mit der Zeit wurden verschiedene Verfahren mit unterschiedlichen Stärken und Schwächen in der Störsicherheit, im Wirkungsgrad, in der Verzerrungsfreiheit und in der Breite des Frequenzspektrums entwickelt. Diese Verfahren lassen sich in zwei Klassen, den Amplitudenmodulations- und Phasenwinkelmodulation- Verfahren unterteilen.\\
In diesen Versuch sollen Beispiele beider Verfahrensklassen untersucht werden.
\section{Theoretische Grundlagen}
\subsection{Amplitudenmodulation}
Eine Amplitudenmodulation kann durch eine hochfrequente Trägerschwingung $U_{\text{T}}\left(t\right)$ mit niederfrequenten Modulationssignal $U_{\text{M}}\left(t\right)$ erreicht werden.\\
Die zeitliche Entwicklung der Amplituden lassen sich durch
\begin{align}
U_{\text{T}}\left(t\right)=\hat{U}_{\text{T}} \cos \omega_{\text{T}}t  \text{ und }U_{\text{M}}\left(t\right)=\hat{U}_{\text{M}} \cos \omega_{\text{M}}t
\end{align}
darstellen. Dabei sind die $\omega_{i}$ jeweils die Frequenzen und $\hat{U}_{i}$ die Amplituden der Trägerschwingung bzw. der Modulationsschwingung.\\
Daraus folgt für die amplitudenmodulierte Schwingung
\begin{align}
U_3\left(t\right)=\hat{U}_{\text{T}} \left(1+m\cos \omega_{\text{M}}t\right)\cos \omega_{\text{T}}t
\label{eq:amplitudederamplitudenmoduliertenschwingung}
\end{align}
mit den Modulationsgrad genannten Größe
\begin{align}
m=\gamma \hat{U}_{\text{M}} \hspace{0.5cm}\text{.}
\end{align}
Der Modulationsgrad kann nur Werte zwischen 0 und 1 annehmen, sodass die Maxima der modulierten Schwingung zwischen $\hat{U}_{\text{T}}\left(1-m\right)$ und $\hat{U}_{\text{T}}\left(1+m\right)$ liegen, wie in Abbildung 
\ref{fig:zeitabhaengigkeit_momentanspannung} anschaulich dargestellt.
Die Gleichung \ref{eq:amplitudederamplitudenmoduliertenschwingung} lässt sich durch trigonometrischer Beziehungen in die Form
\begin{align}
U_3\left(t\right)=\hat{U}_{\text{T}}\left(\cos \omega_{\text{T}}t+ \frac{m}{2}\cos \left( \omega_{\text{T}}+\omega_{\text{M}}\right)t +\frac{m}{2}\cos \left( \omega_{\text{T}}-\omega_{\text{M}}\right)t  \right) 
\label{eq:amplitudederamplitudenmoduliertenschwingungbesser}
\end{align}
überführen, anhand jener das Frequenzspektrum einfach abgelesen werden kann. Das Spektrum besteht aus 3 Linien bei den Frequenzen $\omega_{\text{T}}$, $\omega_{\text{T}}+\omega_{\text{M}}$ und $\omega_{\text{T}}-\omega_{\text{M}}$ und ist Beispielhaft in Abbildung \ref{fig:frequenzsprektum_amplitudenmodulierten} dargestellt.\\
Besteht die Modulationsspannung $\hat{U}_\text{M}$ aus einer Reihe von verschiedener Frequenzen, so verbreitern sich die beiden äußeren Linien zu Frequenzbändern.\\
\begin{figure}[H]
\centering 
\includegraphics[width=10cm]{images/zeitabhaengigkeit_momentanspannung.png}
\caption{Zeitliche Darstellung der Momentanspannung $U_3$ nach Gleichung \ref{eq:amplitudederamplitudenmoduliertenschwingung} (1)}
\label{fig:zeitabhaengigkeit_momentanspannung}
\end{figure}
In Gleichung \ref{eq:amplitudederamplitudenmoduliertenschwingungbesser} ist ersichtlich, dass das Signal bei $\omega_\text{T}$ keinerlei Informationen besitzt, da sie unabhängig von $\hat{U}_\text{M}$ ist. Es wird auch als Trägerabstrahlung bezeichnet und stellt in der Praxis nur einen unnötigen Energieverbrauch da und wird deswegen vermieden.\\
Es kann außerdem Energie gespart werden, indem eines der beiden Bänder durch ein Bandfilter unterdrückt wird. Dies wird auch Einseitenbandmodulation genannt. Es ist möglich, da die beiden Bänder die selbe Information enthalten.
\begin{figure}[H]
\centering 
\includegraphics[width=10cm]{images/frequenzsprektum_amplitudenmodulierten.png}
\caption{Frequenzspektrum einer nach \ref{eq:amplitudederamplitudenmoduliertenschwingungbesser} amplitudenmodulierten Schwingung (1)}
\label{fig:frequenzsprektum_amplitudenmodulierten}
\end{figure}
Die Amplitudenmodulation besitzt jedoch eine geringe Störsicherheit und Verzerrungsfreiheit.
\subsection{Frequenzmodulation} 
Bei der Frequenzmodulation wird nicht die Amplitude, sondern die Frequenz im Rhythmus des Modulationssignales geändert.\\
Es ergibt sich ein Signal der Form
\begin{align}
U\left(t\right)=\hat{U}\sin \left( \omega_{\text{T}}t+ m \frac{\omega_{\text{T}}}{\omega_{\text{M}}}\cos \omega_{\text{M}}t \right)\label{eq:frequenzmodulationsignal}
\end{align}
mit den Modulationsgrad $m$, der durch die Modulationsspannung gegeben ist.\\
Durch Ableitung des Arguments der Sinus-Funktion in \ref{eq:frequenzmodulationsignal} kann die Momentanfrequenz
\begin{align}
f\left(t\right)=\frac{\omega_{\text{T}}}{2\pi}\left(1+m\sin \omega_{\text{M}}t\right)
\end{align} 
bestimmt werden.\\
Des weiteren wird die Größe $m\frac{\omega_{\text{T}}}{2\pi}$ als Frequenzhub bezeichnet und gibt die Variationsbreite der Schwingungsfrequenz an.\\
Wie ein frequenzmoduliertes Signal aussehen könnte, ist in Abbildung
\ref{fig:frequenzmodulation_beispiel} abgebildet.
\begin{figure}[H]
\centering 
\includegraphics[width=13cm]{images/frequenzmodulation_bsp.png}
\caption{Beispielhafter zeitlicher Verlauf eines frequenzmodulierten Signals (1)}
\label{fig:frequenzmodulation_beispiel}
\end{figure} 
Es soll im folgenden nur die Schmalband-Frequenzmodulation betrachtet werden, mit niedrigen Frequenzhub, sodass
\begin{align}
m\frac{\omega_{\text{T}}}{\omega_{\text{M}}}\ll 1 \label{eq:frequenzmodulationsignalentwicklungsparameter}
\end{align}
ist und \ref{eq:frequenzmodulationsignal} nach dieser Größe entwickelt werden kann. Dazu wird \ref{eq:frequenzmodulationsignal} zunächst durch Additionstheoreme in die Form
\begin{align}
U\left(t\right)=\hat{U}\left(\sin   \omega_{\text{T}}t \cos\left(m \frac{\omega_{\text{T}}}{\omega_{\text{M}}}\cos \omega_{\text{M}}t \right)+\cos   \omega_{\text{T}}t \sin\left(m \frac{\omega_{\text{T}}}{\omega_{\text{M}}}\cos \omega_{\text{M}}t \right) \right)\label{eq:frequenzmodulationsignalentwicklungsform}
\end{align}
gebracht, da diese sich leichter nach \ref{eq:frequenzmodulationsignalentwicklungsparameter} entwickeln lässt.\\
Die entstehende Gleichung
\begin{align}
U\left(t\right)=\hat{U}\left(\sin   \omega_{\text{T}}t+ \frac{m}{2}\frac{\omega_{\text{T}}}{\omega_{\text{M}}}\cos\left(\omega_{\text{T}}+\omega_{\text{M}}\right) +\frac{m}{2}\frac{\omega_{\text{T}}}{\omega_{\text{M}}}\cos\left(\omega_{\text{T}}-\omega_{\text{M}}\right)  \right)
\end{align}
hat die gleiche Form wie die Gleichung \ref{eq:amplitudederamplitudenmoduliertenschwingungbesser} beim amplitudenmodulierte Signal und besteht auch aus drei Teilschwingungen, jedoch sind die Seitenlinien um $90^\circ$ gegen die Trägerschwingung in der Phase verschoben.\\
Bei starker Frequenzmodulation mit
\begin{align}
m \omega_{\text{T}} \approx \omega_{\text{M}}
\end{align}
muss die Entwicklung von \ref{eq:frequenzmodulationsignalentwicklungsform} um höhere Ordnungen ergänzt werden.\\
Dabei ergibt sich die Darstellung
\begin{align}
U\left(t\right)=\hat{U}\sum\limits_{n=-\infty}^{\infty} J_n\left(m\frac{\omega_{\text{T}}}{\omega_{\text{M}}} \right)\sin \left(\omega_{\text{T}}+n\omega_{\text{M}}\right)t\label{eq:frequenzmodulation_bessel}
\end{align}
mit der Besselsche Funktion $J_n\left(x\right)$. Dabei wird das Frequenzspektrum in Falle eines hohen Modulationsgrades vollständig abgedeckt, wie es aus Gleichung \ref{eq:frequenzmodulation_bessel} hervorgeht. In praktischen Anwendungen müssen jedoch nur Frequenzen nahe $\omega_{\text{T}}$ betrachtet werden, da die Besselsche Funktion für $x\l 1$ schnell abfällt.   
\section{Durchführung}

\section{Auswertung}

\subsection{Amplitudenmodulation mit Ringmodulator}
Mit einem Ringmodulator wird ein amplitudenmoduliertes Signal mit Trägerunterdrückung erzeugt. Freequenzen f und Amplituden U von Träger- und Modulationssignal lauten:
\begin{enumerate}
	\item Trägersignal f$_t10,0$\,MHz
	\item Modulationssignal f$_m=158$\,kHz
\end{enumerate}


\subsection{Amplitudendemodulation mit Ringmodulator}

\subsection{Amplitudenmodulation mit Diode}

\subsection{Amplitudendemodulation mit Diode}

\subsection{Frequenzmodulation}

\subsection{Frequenzdemodulation}

\section{Diskussion}

\section{Quellen}
(1) Versuchsbeschreibung
\end{document}