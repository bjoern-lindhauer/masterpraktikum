\documentclass[captions=tableheading]{scrartcl}

\usepackage{amsmath}
\usepackage{amssymb}
\usepackage[utf8]{inputenc}
\usepackage[T1]{fontenc}
\usepackage{lmodern}
\usepackage{ngerman}
\usepackage{geometry}
\usepackage{graphicx}
\usepackage{wrapfig}
\usepackage{caption}
\usepackage{wasysym}
\usepackage[separate-uncertainty=true]{siunitx}
\usepackage{picinpar}
\usepackage{tikz}
\usepackage{float}
\usepackage{booktabs}

\renewcommand{\figurename}{Abb.}
\usepackage[
	colorlinks=true,
	urlcolor=blue,
	linkcolor=black
]{hyperref}


%Hier Titel und so
\newcommand{\versuchnummer}{V60} 
\newcommand{\versuchname}{Der Diodenlaser} 
\newcommand{\versuchdatum}{23.01.2017} 


\title{Versuch \versuchnummer\\ \versuchname}
\subtitle{Physikalisches Fortgeschrittenenpraktikum}
\author{Robert Rauter und Björn Lindhauer}
\date{\versuchdatum} 
\begin{document}
\begin{titlepage}
{\large \versuchdatum}
\vspace{7cm}
\begin{center}
\textbf{\huge Versuch \versuchnummer}\\\vspace{0.5cm}
\textbf{\huge \versuchname}\\
\vspace{0.2cm}
\textbf{Physikalisches Fortgeschrittenenpraktikum}\\
\vspace{9cm}

{\Large Robert Rauter \ \ \hspace{1.5cm} und \hspace{1.5cm} Björn Lindhauer}\\
{ \url{robert.rauter@tu-dortmund.de} \ \ \hspace{2cm} \url{bjoern.lindhauer@tu-dortmund.de}}
\end{center}
\end{titlepage}
\section{Einleitung}

\section{Theoretische Grundlagen}

\section{Aufbau}

\section{Durchführung}

\subsection{Justierung des Strahlengangs}
Mithilfe einer Kamera kann das Verhalten des Strahls bei Variation der Stromstärke beobachtet werden. Unterhalb des Grenzwertes der Stomstärke verhält sich der Diodenlaser wie eine LED, d.h. es kommt zu spontaner Emission. Ein Bild es enstehenden Signals ist in Abbildung \ref{fig:led} zu sehen.
\begin{center}
	\includegraphics[width=10cm]{images/led_spot.jpg}
	\captionof{figure}{Beobachtetes Ausgangssignal des Diodenlasers unterhalb der Grenzstromstärke}
	\label{fig:led}
\end{center}
Wird der Grenzwert der Stromstärke überschritten, tritt eine deutliche Erhöhung der Strahlintensität auf, wie es in Abbildung \ref{fig:over_threshold} zu sehen ist. 
\begin{center}
	\includegraphics[width=10cm]{images/over_threshold.jpg}
	\captionof{figure}{Beobachtetes Ausgangssignal des Diodenlaser oberhalb der Grenzstromstärke}
	\label{fig:over_threshold}
\end{center}

\subsection{Beobachtung der Rubidium-Resonanz}

\begin{center}
	\includegraphics[width=10cm]{images/resonanz_bild.jpg}
	\captionof{figure}{Mit Kamera detektiertes Resonanzsignal}
	\label{fig:resonanzbild}
\end{center}

\begin{center}
	\includegraphics[width=10cm]{images/scope_22.png}
	\captionof{figure}{Resonanzspektrum von Rubidium}
	\label{fig:resonanz}
\end{center}

\subsection{Bereinigung des Resonanzsignals}

\begin{center}
	\includegraphics[width=10cm]{images/resonanz.png}
	\captionof{figure}{Resonanzlinien der Rubidium-Probe, mit Hintergrund-Intensität}
	\label{fig:resonanzlinien}
\end{center}

\begin{center}
	\includegraphics[width=10cm]{images/resonanz_bereinigt.png}
	\captionof{figure}{Um Hintergrund bereinigte Resonanzlinien der Rubidium-Probe}
	\label{fig:resonanzlinien_mh}
	\end{center}

\section{Diskussion}

\section{Quellen}
{[1]} Physikalisches Praktikum, TU Dortmund: \\
\textit{Versuchsanleitung zu Versuch 60: Der Diodenlaser} \\
http://129.217.
224.2/HOMEPAGE/PHYSIKER/MASTER/SKRIPT/V60.pdf (letzte Version vom 24.01.2017, 16:00)\\

\end{document}