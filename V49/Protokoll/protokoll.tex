\documentclass[]{scrartcl}

\usepackage{amsmath}
\usepackage{amssymb}
\usepackage[utf8]{inputenc}
\usepackage[T1]{fontenc}
\usepackage{lmodern}
\usepackage{ngerman}
\usepackage{geometry}
\usepackage{graphicx}
\usepackage{wrapfig}
\usepackage{caption}
\usepackage{wasysym}
\usepackage{siunitx}
\usepackage{picinpar}
\usepackage{tikz}
\usepackage{float}

\renewcommand{\figurename}{Abb.}
\usepackage[
	colorlinks=true,
	urlcolor=blue,
	linkcolor=black
]{hyperref}


%Hier Titel und so
\newcommand{\versuchnummer}{V49} 
\newcommand{\versuchname}{Messung von Diffusionskonstanten mittels gepulster Kernspinresonanz} 
\newcommand{\versuchdatum}{11.01.2016} 


\title{Versuch \versuchnummer\\ \versuchname}
\subtitle{Physikalisches Fortgeschrittenenpraktikum}
\author{Robert Rauter und Björn Lindhauer}
\date{\versuchdatum} 
\begin{document}
\begin{titlepage}
{\large \versuchdatum}
\vspace{7cm}
\begin{center}
\textbf{\huge Versuch \versuchnummer}\\
\vspace{0.5cm}
\textbf{\huge \versuchname}\\
\vspace{0.2cm}
\textbf{ Physikalisches Fortgeschrittenenpraktikum}\\
\vspace{9cm}

{\Large Robert Rauter \ \ \hspace{1.5cm} und \hspace{1.5cm} Björn Lindhauer}\\
{ \url{robert.rauter@tu-dortmund.de} \ \ \hspace{2cm} \url{bjoern.lindhauer@tu-dortmund.de}}
\end{center}
\end{titlepage}
\section{Einleitung}

\section{Theoretische Grundlagen}
- Kernspinresonanz: magnetische Momente der Atomkerne richten sich bei äußerem Magnetfeld aus
- Änderung der Ausrichtung durch Einstrahlung von Hochfrequenzquanten mit geeigneter Frequenz veränderbar
a) Resonanzphänomene: Energieaufnahme in Abhängigkeit der Frequenz --> Rückschlüsse auf lokale Magnetfelder durch Resonanzstellen --> Auflösung der Struktur
b) zeitlicher Verlauf von Auf- und Abbau eines Magnetfeldes --> benötigt Hochfrequenzimpulse --> Daher gepulste Kernspinresonanz + durch Diffusion ändert sich \#Momente --> statisches Magnetfeld der Probe ändert sich --> Resonanzbedingung nicht mehr erfüllt  --> Diffusionskonstante bestimmbar


Magnetisierung einer Probe, die im thermischen Gleichgewicht mit der Umgebung steht
Durch anlegen von Feld $B_0\ \vec{e_z}$ spalten die entarteten Kernspinzustände mit der Spinquantenzahl $I$ in $2I+1$ äquidistante Unterniveaus auf.
Besitzen Abstand $\Delta E = \gamma B_0 \hbar$
Niveaus besetzt nach Boltzmann-Verteilung --> Besetzungszahlverhältnis bei $T$ gegeben durch
\begin{align*}
\frac{N\left(m\right)}{N\left(m-1\right)}=\exp \left(-\frac{\gamma B_0\hbar}{k_\text{B}T}\right)
\end{align*}
Da die Besetzung folglich nicht homogen über alle Zustände ist, besitzt der Kern eine Kernspinpolarisation
\begin{align}
\left\langle I_Z \right\rangle = \frac{\sum\limits_{m=-I}^{I} \hbar m \exp\left(-\beta m\gamma B_0 \hbar \right)}{\sum\limits_{m=-I}^{I} \exp\left(-\beta m\gamma B_0 \hbar \right)}
\end{align}
Im folgenden werden nur noch Protonen betrachtet, es ist somit $I=0.5$. Ein Niveau spaltet sich in zwei Unterniveaus auf mit den Quantenzahlen $m=-0.5$ und $m=-0.5$
\section{Durchführung}

\section{Quellen}

\end{document}