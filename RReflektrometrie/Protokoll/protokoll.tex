\documentclass[captions=tableheading]{scrartcl}

\usepackage{amsmath}
\usepackage{amssymb}
\usepackage[utf8]{inputenc}
\usepackage[T1]{fontenc}
\usepackage{lmodern}
\usepackage{ngerman}
\usepackage{geometry}
\usepackage{graphicx}
\usepackage{wrapfig}
\usepackage{caption}
\usepackage{wasysym}
\usepackage[separate-uncertainty=true]{siunitx}
\usepackage{picinpar}
\usepackage{tikz}
\usepackage{float}
\usepackage{booktabs}
\usepackage{enumitem} 

\renewcommand{\figurename}{Abb.}
\usepackage[
	colorlinks=true,
	urlcolor=blue,
	linkcolor=black
]{hyperref}


%Hier Titel und so
\newcommand{\versuchnummer}{V51} 
\newcommand{\versuchname}{Röntgenreflektrometrie} 
\newcommand{\versuchdatum}{16.02.2017} 


\title{Versuch \versuchnummer\\ \versuchname}
\subtitle{Physikalisches Fortgeschrittenenpraktikum}
\author{Robert Rauter und Björn Lindhauer}
\date{\versuchdatum} 
\begin{document}
\begin{titlepage}
{\large \versuchdatum}
\vspace{7cm}
\begin{center}
\textbf{\huge Versuch \versuchnummer}\\\vspace{0.5cm}
\textbf{\huge \versuchname}\\
\vspace{0.2cm}
\textbf{Physikalisches Fortgeschrittenenpraktikum}\\
\vspace{9cm}

{\Large Robert Rauter \ \ \hspace{1.5cm} und \hspace{1.5cm} Björn Lindhauer}\\
{ \url{robert.rauter@tu-dortmund.de} \ \ \hspace{2cm} \url{bjoern.lindhauer@tu-dortmund.de}}
\end{center}
\end{titlepage}

\section{Theoretische Grundlagen}

Röntgenreflexionsversuch --> elektromagnetische Welle mit Wellenlänge $\lambda= \SIrange{0.1}{10}{\angstrom}$

\section{Aufbau}


\section{Durchführung}


\section{Diskussion}

\section{Quellen}
%\renewcommand{\labelenumi}{\value{enumi}}
\begin{enumerate}[label={[\arabic*]}]
\item \label{q:anleitung} \textbf{Physikalisches Praktikum}, TU Dortmund: \\
\textit{Versuchsanleitung zu Versuch 51: Röntgenreflektometrie} \\
\url{http://e1.physik.tu-dortmund.de/cms/Medienpool/Downloads/Roentgenreflektometrie_Versuch.pdf} (letzte Version vom 15.02.2017, 12:26)
\end{enumerate}


\end{document}