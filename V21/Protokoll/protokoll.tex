\documentclass[]{scrartcl}

\usepackage{amsmath}
\usepackage{amssymb}
\usepackage[utf8]{inputenc}
\usepackage[T1]{fontenc}
\usepackage{lmodern}
\usepackage{ngerman}
\usepackage{geometry}
\usepackage{graphicx}
\usepackage{wrapfig}
\usepackage{caption}
\usepackage{wasysym}
\usepackage{siunitx}
\usepackage{picinpar}
\usepackage{tikz}
\usepackage{float}
\usepackage{mathtools}


\renewcommand{\figurename}{Abb.}
\usepackage[
	colorlinks=true,
	urlcolor=blue,
	linkcolor=black
]{hyperref}


%Hier Titel und so
\newcommand{\versuchnummer}{V21} 
\newcommand{\versuchname}{Optisches Pumpen} 
\newcommand{\versuchdatum}{04.07.2016} 


\title{Versuch \versuchnummer\\ \versuchname}
\subtitle{Physikalisches Fortgeschrittenenpraktikum}
\author{Robert Rauter und Björn Lindhauer}
\date{\versuchdatum} 
\begin{document}
\begin{titlepage}
{\large \versuchdatum}
\vspace{7cm}
\begin{center}
\textbf{\huge Versuch \versuchnummer}\\
\vspace{0.5cm}
\textbf{\huge \versuchname}\\
\vspace{0.2cm}
\textbf{ Physikalisches Fortgeschrittenenpraktikum}\\
\vspace{9cm}

{\Large Robert Rauter \ \ \hspace{1.5cm} und \hspace{1.5cm} Björn Lindhauer}\\
{ \url{robert.rauter@tu-dortmund.de} \ \ \hspace{2cm} \url{bjoern.lindhauer@tu-dortmund.de}}
\end{center}
\end{titlepage}
\section{Einleitung}
In diesen Versuch sollen die Landéschen $g$-Faktoren, der Spins der Elektronenhülle und der Spin des Kerns der stabilen Rubidium-Isotope $\prescript{85}{}{\text{Rb}}$ und $\prescript{87}{}{\text{Rb}}$ bestimmt werden.
\section{Theoretische Grundlagen}
Nach dem Aufbauprinzip von Bohr besitzt jedes Atom Elektronenhüllen, welche definierte Energieniveaus besitzen. Es werden zunächst die inneren Niveaus unter der Berücksichtigung des Paul-Prinzips voll besetzt. Ist die äußerste Schale nicht voll besetzt und befindet sich das Atom im thermischen Gleichgewicht, so folgt die Besetzung der Niveaus der äußeren Schale einer Boltzmann-Verteilung gemäß der statistischen Physik.\\
Es lässt sich somit das Besetzungsverhältnis zweier Zustände mit den Energien $W_1$ und $W_2$ durch
\begin{align}
\frac{N_2}{N_1}=\frac{g_2}{g_1}\frac{\exp\left(-\beta W_2\right)}{\exp\left(-\beta W_2\right)}\label{eq:besetzungsverhaeltnisthermisch}
\end{align}
bestimmen. Die $g_i$ sind dabei statistische Gewichte, die die Zahl der Niveaus mit Energie $W_i$ angibt.
\subsection{Ziel des Optisches Pumpen}
Durch ein Verfahren, welches optische Pumpen genannt wird, werden Abweichungen vom Besetzungsverhältnis \ref{eq:besetzungsverhaeltnisthermisch} erzeugt. Diese nicht-thermische Besetzung geht mit der Zeit zurück in das thermische Gleichgewicht, welches durch \ref{eq:besetzungsverhaeltnisthermisch} gegeben ist. Dabei werden Quanten mit einer Energie
\begin{align}
h\nu= W_2 -W_1
\end{align}
emittiert oder absorbiert.\\
Diese Energie kann mit hoher Präzision ausgemessen werden, auch wenn die Energie der Quanten kleiner als die dominierende Energieskala $h\nu \ll k_\text{B}T$ ist. Dies ist beispielsweise bei Niveauunterschiede durch Hyperfeinstrukturaufspaltung oder bei Zeeman-Aufspaltung durch ein Magnetfeld gegeben.\\
Aus dieser Größe lässt sich sowohl der Landéschen $g$-Faktoren als auch der Spin der Elektronenhülle und des Kerns berechnen.\\
Im folgenden
\subsection{Drehimpuls und magnetisches Moment}
\begin{figure}[H]
\begin{minipage}[t]{0.38\textwidth}
\vspace{0pt}
\centering
\includegraphics[width=\textwidth]{images/veranschaulichungdrehimpuls.png}
\caption{Veranschaulichung der Zusammenhänge zwischen Drehimpulsen und magn. Momenten [1]}
\label{fig:veranschaulichungdrehimpuls}
\end{minipage}
\hfill
\begin{minipage}[t]{0.6\textwidth}
\vspace{0pt}
Der Gesamtdrehimpuls $\vec{J}$ der Elektronenhülle ist mit einem magnetischen Moment $\vec{\mu}_J$ gekoppelt. Der Zusammenhang ist durch
\begin{align}
\vec{\mu}_J=-g_J\mu_B \vec{J}
\end{align}
gegeben. Es ist dabei $\mu_B$ das Bohrsche Magneton und $g_J$ der Landé-Faktor.\\
Da $\vec{\mu}_J$ um die $\vec{J}$-Achse präzediert, sind nur die Komponenten $\vec{\mu}_J  	\parallel  \vec{J}$ relevant, da die senkrechte Komponente sich heraus mittelt. Die Zusammenhänge zwischen den Drehimpulsen werden in Abbildung \ref{fig:veranschaulichungdrehimpuls} veranschaulicht.
Es lässt sich für den Landé-Faktor $g_J$ der Ausdruck
\begin{align}
g_J= \frac{\left(1+g_S\right)\left(J^2+J\right) \left(g_S-1\right)\left(S^2+S-L^2-L\right)}{2J\left(J+1\right)}
\end{align}
durch die Anwendung des Kosinus-Theorems herleiten.\\
Das Anlegen eines Magnetfeldes $\vec{B}$ führt zu einer Wechselwirkung zwischen $\vec{\mu}_J$ und $\vec{B}$ mit der Energie
\begin{align}
U_\text{mag}=-\vec{\mu}_J\dot{\vec{B}} \hspace{0.2cm}\text{.}
\end{align}
\end{minipage}
\end{figure}
Die Wechselwirkungsenergie kann aufgrund der Richtungsquantelung nur diskrete Werte
\begin{align}
U_\text{mag}=M_J g_j\mu_B B \hspace{0.2cm}\text{mit} \hspace{0.2cm}M_J \in  \left[-L,L\right]\in \mathbb{Z}
\end{align}
annehmen.\\
Jedes Energieniveau spaltet folglich in $2J+1$ Unterniveaus auf. Dies wird auch als Zeeman-Effekt bezeichnet.
\subsection{Hyperfeinstruktur}
Besitzt ein Kern einen Kernspin $I\ne 0$, so Spalten die Energieniveaus weiter in die Hyperfeinstruktur auf.\\
Es koppelt der Gesamtdrehimpuls der Elektronen $\vec{J}$ und der Kernspin $\vec{I}$ zu einen Gesamtspin
\begin{align}
\vec{F}=\vec{J}+\vec{I} \hspace{0.2cm}\text{.}
\end{align}
Die dazugehörige Quantenzahl $F$ läuft dabei von $I+J$ bis $\left| I-J\right|$. Der zu $F$ gehörige Landé-Faktor $g_F$ ist durch
\begin{align}
g_F\approx g_J\frac{F\left(F+1\right)+J\left(J+1\right)+I\left(I+1\right)}{2F\left(F+1\right)}
\end{align}
gegeben.\\
In Abbildung \ref{fig:hyperfeinaufspaltungbeispiel} ist die Hyperfeinaufspaltung der Niveaus für ein Alkali-Atoms mit Kernspin $I = 3/2$ in zwei Unterniveaus dargestellt.
\begin{figure}[H]
\centering
\includegraphics[width=\textwidth]{images/aufspaltunghyperfein.png}
\caption{Hyperfeinstruktur- und Zeeman-Aufspaltung der Energieniveaus eines Alkali-Atoms mit Kernspin $I = 3/2$ [1]}
\label{fig:hyperfeinaufspaltungbeispiel}
\end{figure}
\section{Durchführung}

\section{Quellen}

\end{document}