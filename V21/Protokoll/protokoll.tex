\documentclass[]{scrartcl}

\usepackage{amsmath}
\usepackage{amssymb}
\usepackage[utf8]{inputenc}
\usepackage[T1]{fontenc}
\usepackage{lmodern}
\usepackage{ngerman}
\usepackage{geometry}
\usepackage{graphicx}
\usepackage{wrapfig}
\usepackage{caption}
\usepackage{wasysym}
\usepackage{siunitx}
\usepackage{picinpar}
\usepackage{tikz}
\usepackage{float}
\usepackage{mathtools}

\renewcommand{\figurename}{Abb.}
\usepackage[
	colorlinks=true,
	urlcolor=blue,
	linkcolor=black
]{hyperref}


%Hier Titel und so
\newcommand{\versuchnummer}{V21} 
\newcommand{\versuchname}{Optisches Pumpen} 
\newcommand{\versuchdatum}{04.07.2016} 


\title{Versuch \versuchnummer\\ \versuchname}
\subtitle{Physikalisches Fortgeschrittenenpraktikum}
\author{Robert Rauter und Björn Lindhauer}
\date{\versuchdatum} 
\begin{document}
\begin{titlepage}
{\large \versuchdatum}
\vspace{7cm}
\begin{center}
\textbf{\huge Versuch \versuchnummer}\\
\vspace{0.5cm}
\textbf{\huge \versuchname}\\
\vspace{0.2cm}
\textbf{ Physikalisches Fortgeschrittenenpraktikum}\\
\vspace{9cm}

{\Large Robert Rauter \ \ \hspace{1.5cm} und \hspace{1.5cm} Björn Lindhauer}\\
{ \url{robert.rauter@tu-dortmund.de} \ \ \hspace{2cm} \url{bjoern.lindhauer@tu-dortmund.de}}
\end{center}
\end{titlepage}
\section{Einleitung}
In diesen Versuch sollen die Landéschen $g$-Faktoren, der Spins der Elektronenhülle und der Spin des Kerns der stabilen Rubidium-Isotope $\prescript{85}{}{\text{Rb}}$ und $\prescript{87}{}{\text{Rb}}$ bestimmt werden.
\section{Theoretische Grundlagen}
Nach dem Aufbauprinzip von Bohr besitzt jedes Atom Elektronenhüllen, welche definierte Energieniveaus besitzen. Es werden zunächst die inneren Niveaus unter der Berücksichtigung des Paul-Prinzips voll besetzt. Ist die äußerste Schale nicht voll besetzt und befindet sich das Atom im thermischen Gleichgewicht, so folgt die Besetzung der Niveaus der äußeren Schale einer Boltzmann-Verteilung gemäß der statistischen Physik.\\
Es lässt sich somit das Besetzungsverhältnis zweier Zustände mit den Energien $W_1$ und $W_2$ durch
\begin{align}
\frac{N_2}{N_1}=\frac{g_2}{g_1}\frac{\exp\left(-\beta W_2\right)}{\exp\left(-\beta W_2\right)}\label{eq:besetzungsverhaeltnisthermisch}
\end{align}
bestimmen. Die $g_i$ sind dabei statistische Gewichte, die die Zahl der Niveaus mit Energie $W_i$ angibt.
\subsection{Optisches Pumpen}
Durch ein Verfahren, welches optische Pumpen genannt wird, werden Abweichungen vom Besetzungsverhältnis \ref{eq:besetzungsverhaeltnisthermisch} erzeugt. Diese nicht-thermische Besetzung geht mit der Zeit zurück in das thermische Gleichgewicht, welches durch \ref{eq:besetzungsverhaeltnisthermisch} gegeben ist. Dabei werden Quanten mit einer Energie
\begin{align}
h\nu= W_2 -W_1
\end{align}
emittiert oder absorbiert.\\
Diese Energie kann mit hoher Präzision ausgemessen werden, auch wenn die Energie der Quanten kleiner als die dominierende Energieskala $h\nu \ll k_\text{B}T$ ist. Dies ist beispielsweise bei Niveauunterschiede durch Hyperfeinstrukturaufspaltung oder bei Zeeman-Aufspaltung durch ein Magnetfeld gegeben.\\
Aus dieser Größe lässt sich sowohl der Landéschen g-Faktoren als auch der Spin der Elektronenhülle und des Kerns berechnen.
\section{Durchführung}

\section{Quellen}

\end{document}